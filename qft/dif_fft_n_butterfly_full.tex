\documentclass[margin=0pt]{standalone} % svg
% \documentclass[convert, margin=0pt]{standalone} % png, don't forget to change latex compiler

\usepackage{anyfontsize, amssymb, amsmath}
% \usepackage[shortlabels]{enumitem}
\usepackage[siunitx, american, RPvoltages]{circuitikz}
\sisetup{parse-numbers=false}
\usepackage{pgfplots}
\pgfplotsset{compat=1.18}
\usepackage[many]{tcolorbox}

% short-hand for resistor units, e.g. 
% in circuitikz: 1<\MO>, 2<\mA>, 3<\kV>, 4<\uF>, 5<\nH>, 6<\C> 
% in math-mode: \qty{1}{\MO}
\DeclareSIUnit{\mO}{\milli\ohm}
\DeclareSIUnit{\kO}{\kilo\ohm}
\DeclareSIUnit{\MO}{\mega\ohm}

\usepackage{xcolor}
% \color{white}

\begin{document}
\def\n{3}
\def\N{\fpeval{2^(\n)}}
\def\ds{1.5}
\def\dh{4.5}
\def\dy{-1.5}
\def\modulo#1#2{\fpeval{trunc(#1-(#2*trunc(#1/#2,0)),0)}}
\def\getbit#1#2{\fpeval{trunc(#1/2^(#2)-(2*trunc(#1/2^(#2+1),0)),0)}}
\begin{circuitikz}
    % \ctikzset{bipole current style/.style={font=\scriptsize}}
    \foreach\j in {0,...,\fpeval{\N-1}}{
            \xdef\revj{0}
            \xdef\xsub{}
            \foreach\stage in {1,...,\n}{
                    \xdef\revj{\fpeval{\revj+\getbit{\j}{\stage-1}*2^(\n-\stage)}}
                    \xdef\xsub{\xsub\ifnum\getbit{\j}{\n-\stage}=0A\else B\fi}
                    \draw \fpeval{((\stage-1)*(\ds+\dh),\dy*\j)}
                    \ifnum\stage=1 node[left]{$x(\j)$}\fi
                    node[ocirc]{}
                    -- ++(\dh*3/4,0)
                    to[short, i_={\ifnum\getbit{\j}{\n-\stage}=1\color{red}$\frac{-1}{\sqrt2}$\else$\frac{1}{\sqrt2}$\fi}, -o]
                    ++(\dh/4,0)
                    to[short, i_={\ifnum\getbit{\j}{\n-\stage}=1$\omega_{\ifnum\stage=1N\else\frac{N}{\fpeval{2^(\stage-1)}}\fi}^{\modulo{\j}{2^(\n-\stage)}}$\else~\fi}]
                    ++(\ds,0)
                    node[ocirc]{}
                    node[above]{\color{black!20}\tiny$x_{\xsub}(\modulo{\j}{2^(\n-\stage)})$}
                    ;
                    \draw \fpeval{((\stage-1)*(\ds+\dh),\dy*\j)}
                    \ifnum\getbit{\j}{\n-\stage}=0
                    edge[-Latex] ++(\dh,\fpeval{\dy*2^(\n-\stage)})
                    \else
                    edge[-Latex] ++(\dh,\fpeval{-\dy*2^(\n-\stage)})
                    \fi
                    ;
                }
            \draw \fpeval{(\n*(\ds+\dh),\dy*\j)}
            node[right=12pt]{$X(\revj)$}
            ;
        }
\end{circuitikz}
\end{document}
