\documentclass[margin=0pt]{standalone} % svg
% \documentclass[convert, margin=0pt]{standalone} % png, don't forget to change latex compiler

\usepackage{anyfontsize, amssymb, amsmath}
% \usepackage[shortlabels]{enumitem}
\usepackage[siunitx, american, RPvoltages]{circuitikz}
\sisetup{parse-numbers=false}
\usepackage{pgfplots}
\pgfplotsset{compat=1.18}
\usepackage[many]{tcolorbox}

% short-hand for resistor units, e.g. 
% in circuitikz: 1<\MO>, 2<\mA>, 3<\kV>, 4<\uF>, 5<\nH>, 6<\C> 
% in math-mode: \qty{1}{\MO}
\DeclareSIUnit{\mO}{\milli\ohm}
\DeclareSIUnit{\kO}{\kilo\ohm}
\DeclareSIUnit{\MO}{\mega\ohm}

\usepackage{xcolor}
% \color{white}

\begin{document}
\def\n{3}
\def\N{\fpeval{2^(\n)}}
\def\ds{1}
\def\dh{3}
\def\dy{-1}
\def\modulo#1#2{\fpeval{trunc(#1-(#2*trunc(#1/#2,0)),0)}}
\def\getbit#1#2{\fpeval{trunc(#1/2^(#2)-(2*trunc(#1/2^(#2+1),0)),0)}}
\begin{circuitikz}
    \foreach\j in {0,...,\fpeval{\N-1}}{
        \xdef\revj{0}
        \foreach\stage in {1,...,\n}{
            \draw \fpeval{((\stage-1)*(\ds+\dh),\dy*\j)}
            \ifnum\stage=1 node[left]{$x(\j)$}\fi
            node[ocirc]{}
            -- ++(\dh*3/4,0)
            to[short, i_={\scriptsize\ifnum\getbit{\j}{\n-\stage}=1$-1$\else~\fi}, -o]
            ++(\dh/4,0)
            to[short, i_={\scriptsize\ifnum\getbit{\j}{\n-\stage}=1$\omega_{N\ifnum\stage>1/\fpeval{2^(\stage-1)}\fi}^{\modulo{\j}{2^(\n-\stage)}}$\else~\fi}, -o]
            ++(\ds,0)
            ;
            \draw \fpeval{((\stage-1)*(\ds+\dh),\dy*\j)}
            \ifnum\getbit{\j}{\n-\stage}=0
            edge[-Latex] ++(\dh,\fpeval{\dy*2^(\n-\stage)})
            \else
            edge[-Latex] ++(\dh,\fpeval{-\dy*2^(\n-\stage)})
            \fi
            ;
            \xdef\revj{\fpeval{\revj+\getbit{\j}{\stage-1}*2^(\n-\stage)}}
        }
        \draw \fpeval{(\n*(\ds+\dh),\dy*\j)}
        node[right]{$X(\revj)$}
        ;
    }
\end{circuitikz}
\end{document}
