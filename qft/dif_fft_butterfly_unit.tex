\documentclass[margin=0pt]{standalone} % svg
% \documentclass[convert, margin=0pt]{standalone} % png, don't forget to change latex compiler

\usepackage{anyfontsize, amssymb, amsmath}
% \usepackage[shortlabels]{enumitem}
\usepackage[siunitx, american, RPvoltages]{circuitikz}
\sisetup{parse-numbers=false}
\usepackage{pgfplots}
\pgfplotsset{compat=1.18}
\usepackage[many]{tcolorbox}

% short-hand for resistor units, e.g. 
% in circuitikz: 1<\MO>, 2<\mA>, 3<\kV>, 4<\uF>, 5<\nH>, 6<\C> 
% in math-mode: \qty{1}{\MO}
\DeclareSIUnit{\mO}{\milli\ohm}
\DeclareSIUnit{\kO}{\kilo\ohm}
\DeclareSIUnit{\MO}{\mega\ohm}

\usepackage{xcolor}
% \color{white}

\begin{document}
\begin{circuitikz}
    \draw (0,0)
    to[short, i=$\frac{1}{\sqrt2}$, inner sep=0pt, current/distance=0.85] (3,-2)
    (0,-2)
    to[short, i_=$\frac{1}{\sqrt2}$, inner sep=0pt, current/distance=0.85] (3,0)
    ;
    \draw (0,0)
    node[left]{$x(j)$}
    node[ocirc]{}
    to[short, i^=$\frac{1}{\sqrt2}$, current/distance=0.85] ++(3,0)
    node[adder, scale=0.25, fill=white]{}
    to[short, i_=~, -o] ++(1,0)
    node[right]{$x_A(j)$}
    ;
    \draw (0,-2)
    node[left]{$x\left(j+\frac N2\right)$}
    node[ocirc]{}
    to[short, i_=$\frac{-1}{\sqrt2}$, current/distance=0.85] ++(3,0)
    node[adder, scale=0.25, fill=white]{}
    to[short, i_=$\omega_N^j$, -o] ++(1,0)
    node[right]{$x_B(j)$}
    ;
\end{circuitikz}
\end{document}
